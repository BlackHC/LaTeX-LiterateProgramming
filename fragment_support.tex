%	Copyright 2013 Andreas Kirsch
%	
%	Permission is hereby granted, free of charge, to any person obtaining
%	a copy of this software and associated documentation files (the
%	"Software"), to deal in the Software without restriction, including
%	without limitation the rights to use, copy, modify, merge, publish,
%	distribute, sublicense, and/or sell copies of the Software, and to
%	permit persons to whom the Software is furnished to do so, subject to
%	the following conditions:
%	
%	The above copyright notice and this permission notice shall be
%	included in all copies or substantial portions of the Software.
%	
%	THE SOFTWARE IS PROVIDED "AS IS", WITHOUT WARRANTY OF ANY KIND,
%	EXPRESS OR IMPLIED, INCLUDING BUT NOT LIMITED TO THE WARRANTIES OF
%	MERCHANTABILITY, FITNESS FOR A PARTICULAR PURPOSE AND
%	NONINFRINGEMENT. IN NO EVENT SHALL THE AUTHORS OR COPYRIGHT HOLDERS BE
%	LIABLE FOR ANY CLAIM, DAMAGES OR OTHER LIABILITY, WHETHER IN AN ACTION
%	OF CONTRACT, TORT OR OTHERWISE, ARISING FROM, OUT OF OR IN CONNECTION
%	WITH THE SOFTWARE OR THE USE OR OTHER DEALINGS IN THE SOFTWARE.

\usepackage{fixltx2e}

% for \carriagereturn
\usepackage{dingbat}

%% Load float package, for enabling floating extensions
\usepackage{float}

\newfloat{listing}{tbph}{lst}[chapter]
\floatname{listing}{Listing}

\usepackage{etextools}
\usepackage{xstring}
\usepackage{xparse}

\usepackage{listings}

%%%%%%%%%%%%%%%%%%%%%%%%%%%%%%%%%%%%%%%%%%%%%%%%%%%%%%%%%%%%%%%%%%%%%%
%% Code fragment support

\lstset{language=C++,tabsize=2}
\lstset{mathescape=true,escapechar=�,texcl=false}

\newcommand{\listingsUseFancyLineBreaks}{%
	\lstset{%
	%	prebreak={\raisebox{-1ex}[0ex][0ex]{\carriagereturn}},%
	%	postbreak={\raisebox{0ex}[0ex][0ex]{\ensuremath{\rcurvearrowse\space}}},%
		breaklines,%
		breakatwhitespace,%
		breakindent=10pt,%
		breakautoindent=true%
	}%
}

\newcommand{\listingsUseFancyFonts}{%
	\newcommand{\myListingBasicStyle}{\ttfamily\fontsize{9pt}{10pt}\selectfont}%
	\lstset{columns=flexible,basicstyle={\myListingBasicStyle},fontadjust=true}%
}
\listingsUseFancyFonts

\newcommand{\listingsSpacing}{%
	\lstset{abovecaptionskip=0pt,belowcaptionskip=0pt}%
	\lstset{aboveskip=\smallskipamount,belowskip=\medskipamount}%
}

\lstMakeShortInline[breaklines,breakindent=0pt,breakautoindent=false,breakatwhitespace]�

\lstnewenvironment{fancylisting}{%
	\listingsSpacing%
	\listingsUseFancyLineBreaks%
}{%
	\ignorespacesafterend%
}

\newcommand{\createFragmentLabel}[1]{%
  \lowercase{\def\fragmentLabelName{#1}}%
  \StrSubstitute{\fragmentLabelName}{,}{-}[\fragmentLabelName]%
  \StrSubstitute{\fragmentLabelName}{ }{-}[\fragmentLabelName]%
  \StrSubstitute{\fragmentLabelName}{\ }{-}[\fragmentLabelName]%
  %\fragmentLabelName%
}

\newcommand{\createFragmentFileLabel}[1]{%
  %\lowercase{\def\fragmentFileLabel{#1}}%
  \def\fragmentFileLabel{#1}%
  \StrSubstitute{\fragmentFileLabel}{ }{\ }[\fragmentFileLabel]%
  \StrSubstitute{\fragmentFileLabel}{-}{..}[\fragmentFileLabel]%
}

\newcommand*{\putFragmentName}[1]{{\myListingBasicStyle$\langle$\textit{#1}$\rangle$}}
\newcommand*{\putFragmentRefs}[1]{{\textnormal{\bfseries \scriptsize #1}}}
\lstnewenvironment{fragment}
	[2][]
	{%
%		\begingroup%
		\createFragmentLabel{fragment:#2}%
		\listingsUseFancyLineBreaks%
		\listingsSpacing%
		\lstset{title={\putFragmentName{#2}\label{\fragmentLabelName}#1$\equiv$\hspace*{\fill}}}%
%		\endgroup%
	}
	{}

% {prefix}{name}{file}{fileLabel}{refLabel}
\newcommand*{\fileFragmentHelper}[5]{%
	%{%
	\listingsUseFancyLineBreaks%
	\listingsSpacing%
	\lstinputlisting[%
		rangebeginprefix=//<,rangebeginsuffix=>\ =,%
		rangeendprefix=//<,rangeendsuffix=>,%
		includerangemarker=false,%
		linerange={#4}-end,%
		title={\putFragmentName{#2}#1$\equiv$\hspace*{\fill}},%
		label=#5%
	]{#3}%
	%}%
	%\ignorespacesafterend%
}

% {fragment}{file}
\newcommand*{\fileFragment}[2]{%
	%\begingroup%
	\createFragmentLabel{fragment:#1}%
	\createFragmentFileLabel{#1}%
	%\\refLabel:\fragmentLabelName\\fileLabel:\fragmentFileLabel
	\expandnexttwo{\fileFragmentHelper{}{#1}{#2}}{\fragmentFileLabel}{\fragmentLabelName}%
	%\endgroup%
}

% {prefix}{name}{tag}{file}
\newcommand*{\taggedFileFragment}[4]{%
	%\begingroup%
	\createFragmentLabel{fragment:#2-#3}%
	\createFragmentFileLabel{#2;#3}%
	%\\refLabel:\fragmentLabelName\\fileLabel:\fragmentFileLabel
	\expandnexttwo{\fileFragmentHelper{#1}{#2}{#4}}{\fragmentFileLabel}{\fragmentLabelName}%
	%\endgroup%
}


% nice syntax
\def\defFileFragment<#1>#2{\fileFragment{#1}{#2}}
\def\defTaggedFileFragment<#1;#2>#3=#4{\taggedFileFragment{#3}{#1}{#2}{#4}}

% even nicer syntax
\newcommand{\setCurrentFragmentFile}[1]{\def\currentFragmentFile{#1}}
\def\defCurrentFileFragment<#1>{\expandnext{\defFileFragment<#1>}{\currentFragmentFile}}
\def\defTaggedCurrentFileFragment<#1;#2>#3={\expandnext{\taggedFileFragment{#3}{#1}{#2}}{\currentFragmentFile}}

% we only want to pageref fragment's that have actually been defined..
\makeatletter
\newcommand{\fragmentSafePageRef}[1]{%
	\begingroup%
		\createFragmentLabel{fragment:#1}%
		%\fragmentLabelName\\
		\expandafter\@ifundefined{r@\fragmentLabelName}{\relax}{ \putFragmentRefs{\pageref{\fragmentLabelName}}}%
    \endgroup%
}

\makeatother

\def\refFragment<#1>{%
	\putFragmentName{#1\fragmentSafePageRef{#1}}%
}
